\documentclass[UTF8]{ctexart}
\usepackage{titlesec}
\title{Horizon-图书馆系统\\用户手册}
\author{陈欣昊,林伟鸿,应思豪,吴怡然,杜泓睿,叶昊然}
\date{\today}
\begin{document}
\maketitle
\newpage
\tableofcontents
\newpage
\section{版权声明}
\subsection{版权声明}
\par
该图书馆系统的著作权,版权和知识产权属于作者六人所有,并受《中华人民共和国著作权法》、《中华人民共和国专利法》、《计算机软件保护条例》、《知识产权保护条例》以及其他知识产权法律和条约的保护。
\par
任何单位和个人未经作者授权不得用于商业用途,否则视为非法侵害,作者保留依法追究其责任的权利。
\section{引言}
\subsection{编写目的}
\par	
应用数字化的管理系统对图书馆书籍以及用户的各类信息进行管理。
\subsection{背景}
\subsubsection{该系统任务提出者}
\par
程序设计老师与助教。
\subsubsection{参考资料}
\par
《C++程序设计思想与方法》。
\subsubsection{使用的软件}
\par
QT 5.5, Visual Studio 2013, Xcode 7, TeXShop。
\section{使用与功能}
\subsection{软件目标}
\par
以精简的界面为载体,使用户们有条不紊的使用或管理图书馆的各类信息。
\subsection{使用之前}
\par
点击登录界面的注册按钮,输入对应信息进行注册。
\subsection{使用者}
\par
共设定了三类使用者:普通用户,管理员与馆长。
\subsubsection{普通用户}
\par
查看书籍信息,借阅书籍,归还书籍,修改自己的用户信息,查看自己的历史操作(日志系统)。
\subsubsection{管理员}
\par
添加书籍,删除书籍,添加用户,删除用户,查询用户信息。管理员也可以进行普通用户的所有操作。
\subsubsection{馆长}
\par
将普通用户设为管理员,将管理员降级为普通用户。馆长也可以进行管理员的所有操作。
\subsection{软件扩展功能}
\subsubsection{数据加密}
\par
采用sha512方法对用户数据进行加密,充分保障用户数据的安全。
\subsubsection{数据压缩}
\par
使用了数据压缩算法,减少所占存储空间。
\subsubsection{通配符搜索}
\par
各类数据采用SQLite数据库存储,支持通配符搜索,且速度极快(10万本书搜索只需0.1秒)。
\subsubsection{跨平台}
\par
经测试,该图书管系统软件可在Windows,Linux与Mac上使用。
\subsubsection{UI}
\par
使用了立绘,使得界面亲切友好。
\end{document}